\chapter{Tổng quan nghiên cứu}
\label{chap:1tqnc}
\section{Giới thiệu robot tự hành thông minh}

% - Thế nào là robot tự hành thông minh
% - ứng dụng của robot tự hành thông minh

\section{Các bài toán trên robot tự hành thông minh}

% - Nêu qua concept về điều khiển robot, tạo bản đồ, định vị, path planning, điều hướng, tránh vật cản

\section{Các nghiên cứu tránh vật cản trong robot tự hành thông minh}
\label{sec:tranhVatCan_ref}

% Review các bài báo về tránh vật cản trong robot tự hành
% Chỉ ra các ưu, nhược điểm của các phương pháp như được nêu trong các bài báo trên

\section{Nội dung nghiên cứu}

% Chốt lại nội dung nghiên cứu gồm 2 nội dung:
% - Điều khiển robot ứng dụng SLAM trên nền tảng hệ điều hành ROS
% - Sử dụng Multi-sensor để tăng cường phát hiện và tránh vật cản cho robot
% - Trình bày vắn tắt nội dung của các chương.

%%===========================
\chapter{Cơ sở lý thuyết}
\label{chap:1cslt}
\section{Bài toán về nhiễu trong robot tự hành}

\section{Bài toán SLAM 2D}

\section{Bài toán tạo định vị, tạo bản đồ, điều hướng và tránh vật cản}

\section{Hệ điều hành robot ROS và các ứng dụng}

% Tóm gọn nội dung giới thiệu về ROS và các ứng dụng trong khuôn khổ của luận văn này.

%% ===================
%%% Local Variables:
%%% mode: latex
%%% TeX-master: "../LuanVanThS_v1.0_main"
%%% End:
